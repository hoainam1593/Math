\documentclass{article}
\usepackage[T5]{fontenc}
\usepackage[utf8]{inputenc}
\usepackage[vietnamese]{babel}
\usepackage{graphicx}

\begin{document}

1. Các số trên tử và dưới mẫu giống nhau vậy kết quả là 1

2. $S=(90 + 99)\times 5=189 \times 5=945$

3. $T=\frac{10^3}{5} =2\times 10 \times 10=200$

4. Ta thấy $S=S(ABCS)-S(DEFS)=6\times 9 - 2\times 4=54-8=46$

\begin{figure}[h]
    \centering
    \includegraphics[height=4cm]{images/solution-4.png}
    \label{fig:sample_4}
\end{figure}

5. Tổng học sinh là $A=5+4+3+3+5=20$ \newline
Số học sinh đậu là $B=5+4+3+3=15$ \newline
Vậy tỉ lệ đậu là $\frac{B}{A}=\frac{15}{20}=\frac{3}{4}$

6. Số lượng tờ = $\frac{7.5}{5} \times 500 = \frac{3}{2} \times 500 = 250\times 3 = 750$

7. Hàng 1 có 0 ô đen, hàng 2 có 1, hàng 3 có 2, .... vậy hàng 37 sẽ có 36 ô đen.

8. số lớn nhất là số dương, vậy số lớn nhất là $-3a, a^2$ hoặc 1. Dễ thấy số lớn nhất là $-3a$.

9. $A=\frac{1}{2}\times \frac{2}{3}\times \frac{3}{4}\dots \frac{9}{10}=\frac{1}{10}$

10. số chính giữa là $( \frac{1}{5}+\frac{1}{3}) \div 2=\frac{8}{15}\div 2=\frac{4}{15}$

11. dễ thấy đó là mặt Y.

12. gọi cạnh hình vuông là a, ta có:
\[
4a=6.2+8.3+9.5=24 \Rightarrow a=6
\]
vậy diện tích hình vuông là $A=a^2=6^2=36$

13. quãng đường đi bộ được là $0.75\times 4=3$ miles. quãng đường chạy được là $0.5\times 10=5$ miles.
vậy tổng quãng đường là $3+5=8$ miles.

\end{document}