\documentclass{article}
\usepackage[T5]{fontenc}
\usepackage[utf8]{inputenc}
\usepackage[vietnamese]{babel}
\usepackage{graphicx}

\begin{document}

1. Tính 
\[
    \frac{3\times 5}{9\times 11}\times \frac{7\times 9\times 11}{3\times 5\times 7} 
\]

2. Tính
\[
    90 + 91 + 92 + 93 + 94 + 95 + 96 + 97 + 98 + 99
\]

3. Tính
\[
    \frac{10^7}{5\times 10^4}
\]

4. Tính diện tích của đa giác ABCDEF

\begin{figure}[h]
    \centering
    \includegraphics[height=4cm]{images/problem-4.png}
    \label{fig:sample_4}
\end{figure}

5. Biểu đồ bên dưới thể hiện số lượng học sinh được điểm A, B, C, D, F trong một kì thi. Biết rằng được điểm A, B, C, D sẽ đậu, tỉ lệ đậu là bao nhiêu?

\begin{figure}[h]
    \centering
    \includegraphics[height=5cm]{images/problem-5.png}
    \label{fig:sample_5}
\end{figure}

6. Một chồng giấy gồm 500 tờ sẽ cao 5cm. Hỏi chồng giấy cao 7.5cm gồm bao nhiêu tờ?

\newpage

7. Nhìn vào hình bên dưới cho biết: Tại hàng thứ 37 có nhiêu ô màu đen?

\begin{figure}[h]
    \centering
    \includegraphics[height=3cm]{images/problem-7.png}
    \label{fig:sample_7}
\end{figure}

8. Nếu $a=-2$, số nào sau đây lớn nhất:
\[
    \{ -3a, 4a, \frac{24}{a}, a^2, 1 \} 
\]


9. Tính

\[
    \left(1-\frac{1}{2} \right) \left(1-\frac{1}{3} \right) \left(1-\frac{1}{4} \right) \dots \left(1-\frac{1}{10} \right) 
\]

10. Số chính giữa hai phân số $\frac{1}{5}$ và $\frac{1}{3}$ là số nào?

11. Hình bên dưới là tờ giấy được cắt và ghép lại thành khối lập phương. Sau khi ghép lại thành khối lập phương, mặt đối diện với mặt X là mặt nào?

\begin{figure}[h]
    \centering
    \includegraphics[height=4cm]{images/problem-11.png}
    \label{fig:sample_11}
\end{figure}

12. Một hình vuông và một hình tam giác có cùng chu vi, ba cạnh tam giác có độ dài là $6.2 cm$, $8.3 cm$, $9.5 cm$, tính diện tích hình vuông.

13. Nếu bạn đi bộ 45 phút với vận tốc $4mph$, sau đó chạy 30 phút với vận tốc $10mph$. Vậy sau 1 giờ 15 phút đó, bạn đã di chuyển bao xa?

14. Một món hàng có giá $20\$$. Sự khác biệt về giá trước thuế giữa thuế $6.5\%$ và $6\%$ là bao nhiêu?

15. Có bao nhiêu số tự nhiên giữa 100 và 400 mà chứa chữ số 2?

16. Trong lớp học toán của thầy Brown, tỉ lệ nam/nữ là $2:3$, biết rằng trong lớp có 30 học sinh, hỏi học sinh nữ nhiều hơn học sinh nam bao nhiêu?

17. Nếu điểm trung bình trong 6 lần kiểm tra toán là 84 và điểm trung bình trong 7 lần kiểm tra toán là 85, vậy trong lần kiểm tra thứ 7, điểm là bao nhiêu?

18. 9 tờ của một loại tờ rơi có giá ít hơn $10\$$ và 10 tờ có giá nhiều hơn $11\$$. Hỏi mỗi tờ giá bao nhiêu? (Làm tròn giá đến 2 chữ số thập phân)

19. Nếu chiều dài và chiều rộng hình chữ nhật tăng $10\%$ thì chu vi tăng bao nhiêu?

20. Trong một năm nào đó, tháng 1 có đúng 4 thứ ba và 4 thứ bảy, hỏi ngày 1/1 là thứ mấy?

21. Ông Green được tăng lương $10\%$ mỗi năm. Sau bốn lần tăng như vậy, lương ổng đã tăng bao nhiêu phần trăm?

22. Giả sử số điện thoại gồm 7 chữ số nhưng không được bắt đầu bằng 0 và 1. Tính tỉ lệ số điện thoại bắt đầu bằng 9 và kết thúc bằng 0 trên tổng số tấc cả các số điện thoại.

23. Trường trung học King có 1200 học sinh. Mỗi học sinh học 5 lớp một ngày. Mỗi giáo viên dạy 4 lớp. Mỗi lớp có 30 học sinh và 1 giáo viên. Hỏi có bao nhiêu giáo viên?

24. Bên dưới là một tam giác phép thuật, mỗi số từ $10-15$ sẽ điền vào mỗi ô tròn sao cho tổng S của ba số trên mỗi cạnh bằng nhau. Giá trị lớn nhất của S là bao nhiêu?

\begin{figure}[h]
    \centering
    \includegraphics[height=4cm]{images/problem-24.png}
    \label{fig:sample_24}
\end{figure}

25. Hình bên dưới là 5 lá bài nằm trên bàn. Mỗi lá bài có một mặt là kí tự còn một mặt là số. Jane nói "Nếu một nguyên âm (a,e,i) là một mặt của bất kì lá bài nào, thì mặt còn lại là số chẵn". Mary đã chứng minh Jane sai bằng cách lật một lá bài. Hỏi Mary đã lật lá nào?

\begin{figure}[h]
    \centering
    \includegraphics[height=2cm]{images/problem-25.png}
    \label{fig:sample_25}
\end{figure}

\end{document}