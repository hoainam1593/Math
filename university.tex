\documentclass{article}
\begin{document}

\section*{2003}
\section*{A}

câu 5: cho $x,y,z$ là các số thực dương thỏa mãn $x+y+z\leq1$. Chứng minh rằng:
\[\sqrt{x^2+\frac{1}{x^2}}+\sqrt{y^2+\frac{1}{y^2}}+\sqrt{z^2+\frac{1}{z^2}}\geq\sqrt{82}\]
lời giải: 
\[x^2+\frac{1}{81x^2}\geq\frac{2}{9}\Rightarrow x^2+\frac{1}{x^2}\geq\frac{2}{9}+\frac{80}{81x^2}\]
\[\Rightarrow \sqrt{x^2+\frac{1}{x^2}}\geq\sqrt{\frac{2}{9}+\frac{80}{81x^2}}\]

dùng bất đẳng thức Bunyakovsky:
\[(a_1^2+a_2^2+...+a_n^2).(b_1^2+b_2^2+...+b_n^2)\geq(a_1.b_1+a_2.b_2+...+a_n.b_n)^2\]
cho $b_1=1, b_2=1, ..., b_n=1$ ta có:
\[n.(a_1^2+a_2^2+...+a_n^2)\geq(a_1+a_2+...+a_n)^2\]
\[\Rightarrow n.(t_1+t_2+...+t_n)\geq(\sqrt{t_1}+\sqrt{t_2}+...+\sqrt{t_n})^2\]
\[\Rightarrow \sqrt{t_1+t_2+...+t_n}\geq\frac{1}{\sqrt{n}}.(\sqrt{t_1}+\sqrt{t_2}+...+\sqrt{t_n})\]

áp dụng bất đẳng thức trên cho $\frac{2}{9}, \frac{2}{81x^2},..., \frac{2}{81x^2}$, ta có:
\[\sqrt{\frac{2}{9}+\frac{80}{81x^2}}\geq\frac{1}{\sqrt{41}}.(\sqrt{\frac{2}{9}}+40.\sqrt{\frac{2}{81x^2}})\]
\[=\frac{1}{\sqrt{41}}.(\frac{\sqrt{2}}{3}+\frac{40\sqrt{2}}{9}.\frac{1}{x})\]
tương tự cho $y,z$, từ đó ta có:
\[P\geq\frac{1}{\sqrt{41}}.(\sqrt{2}+\frac{40\sqrt{2}}{9}.(\frac{1}{x}+\frac{1}{y}+\frac{1}{z}))\]
\[\geq\frac{1}{\sqrt{41}}.(\sqrt{2}+\frac{40\sqrt{2}}{9}.\frac{9}{x+y+z})\geq\sqrt{82}\]

\section*{B}






câu 4:
\newline
1) tìm min, max của hàm số 
\[x+\sqrt{4-x^2}\]

lời giải:
đặt $y=\sqrt{4-x^2}$, khi đó $x\in [ -2,2] , y\in [ 0,2], x^2+y^2=4, f=x+y$
\[f^2=4+2xy\leq 8\Rightarrow f\leq2\sqrt{2}\]

vậy max của f la $2\sqrt{2}$
\newline
giả sử $x\leq0$, khi đó
\[f^2=4+2xy\leq4\]
suy ra min của f là -2 khi $x\leq0$, 
\newline
với $x\geq0$, dễ thấy f luôn lớn hơn 0
\newline
vậy min của f là -2


\section*{D}

\section*{2005}
\section*{A}

câu 5: cho $x,y,z$ là các số dương thỏa mãn $\frac{1}{x}+\frac{1}{y}+\frac{1}{z}=4$, chứng minh rằng:
\[\frac{1}{2x+y+z}+\frac{1}{x+2y+z}+\frac{1}{x+y+2z}\leq1\]

lời giải: áp dụng
\[\frac{1}{a+b+c+d}\leq\frac{1}{16}.(\frac{1}{a}+\frac{1}{b}+\frac{1}{c}+\frac{1}{d})\]
ta có:
\[\frac{1}{2x+y+z}\leq\frac{1}{16}.(\frac{2}{x}+\frac{1}{y}+\frac{1}{z})\]
tương tự cho $y,z$ và cộng từng vế ta được:
\[P\leq\frac{1}{4}.(\frac{1}{x}+\frac{1}{y}+\frac{1}{z})=1\]

\section*{B}


câu 5: chứng minh rằng với mọi $x\in R $ ta có:

\[(\frac{12}{5})^x+(\frac{15}{4})^x+(\frac{20}{3})^x\geq3^x+4^x+5^x\]

lời giải: áp dụng






\[\frac{ab}{c}+\frac{bc}{a}+\frac{ca}{b}\geq a+b+c\]

cho $3^x, 4^x, 5^x$

\section*{D}

câu 5: cho các số dương $x,y,z$ thỏa $xyz=1$, chứng minh rằng:

\[\frac{\sqrt{1+x^3+y^3}}{xy}+\frac{\sqrt{1+y^3+z^3}}{yz}+\frac{\sqrt{1+z^3+x^3}}{zx}\geq 3\sqrt{3}\]

lời giải:


\[1+x^3+y^3\geq3xy\Rightarrow \frac{\sqrt{1+x^3+y^3}}{xy}\geq \frac{\sqrt{3}}{\sqrt{xy}}\]
\[\Rightarrow P\geq \sqrt{3}.(\frac{1}{\sqrt{xy}}+\frac{1}{\sqrt{yz}}+\frac{1}{\sqrt{zx}})\geq3\sqrt{3}\]

\section*{2006}
\section*{A}

câu 4:
\newline
2) cho hai số thực $x,y$ thỏa
\[(x+y)xy=x^2+y^2-xy\]
tìm giá trị lớn nhất của $\frac{1}{x^3}+\frac{1}{y^3}$

lời giải:

đặt $a=\frac{1}{x}, b=\frac{1}{y}$, khi đó:

\[(\frac{1}{a}+\frac{1}{b}).\frac{1}{ab}=\frac{1}{a^2}+\frac{1}{b^2}-\frac{1}{ab}\]
\[\Rightarrow a+b=a^2+b^2-ab\Rightarrow a+b+3ab=(a+b)^2\]

đặt $t=a+b$, khi đó $ab=\frac{t^2-t}{3}$
ta có $(a+b)^2\geq4ab\Rightarrow t^2\geq\frac{4}{3}.(t^2-t)$
\[\Rightarrow 0\leq t \leq 4 \Rightarrow 0\leq a+b \leq 4\]

vậy $P=a^3+b^3=(a^2+b^2)(a+b)-ab(a+b)=(a+b+ab)(a+b)-ab(a+b)=(a+b)^2\leq16$

\section*{B}

câu 4:
\newline
2) cho số thực $x,y$ tìm giá trị nhỏ nhất của

\[\sqrt{(x-1)^2+y^2}+\sqrt{(x+1)^2+y^2}+|y-2| \]

lời giải: áp dụng BĐT Minkowski
\[\sqrt{a^2+b^2}+\sqrt{c^2+d^2}\geq\sqrt{(a+c)^2+(b+d)^2}\]
dấu = xảy ra khi $ad=bc$

\[\sqrt{(x-1)^2+y^2}+\sqrt{(x+1)^2+y^2}\geq 2\sqrt{1+y^2}\]
\[\Rightarrow P\geq 2\sqrt{1+y^2}+|y-2|\]

khảo sát hàm số này, ta thấy min đạt được khi $y=\frac{1}{\sqrt{3}}$
vậy min của P là $2+\sqrt{3}$

\section*{D}

\section*{2007}
\section*{A}

câu 4
\newline
2) cho các số dương $x,y,z$ thỏa $xyz=1$, tìm giá trị nhỏ nhất:

\[\frac{x^2(y+z)}{y\sqrt{y}+2z\sqrt{z}}+\frac{y^2(z+x)}{z\sqrt{z}+2x\sqrt{x}}+\frac{z^2(x+y)}{x\sqrt{x}+2y\sqrt{y}}\]

lời giải: 

\[\frac{x^2(y+z)}{y\sqrt{y}+2z\sqrt{z}}\geq\frac{2x^2.\sqrt{yz}}{y\sqrt{y}+2z\sqrt{z}}\]
\[=\frac{2x\sqrt{x}}{y\sqrt{y}+2z\sqrt{z}}\]
\[\Rightarrow P\geq\frac{2x\sqrt{x}}{y\sqrt{y}+2z\sqrt{z}}+\frac{2y\sqrt{y}}{z\sqrt{z}+2x\sqrt{x}}+\frac{2z\sqrt{z}}{x\sqrt{x}+2y\sqrt{y}}\]

đặt $a=x\sqrt{x},b=y\sqrt{y},c=z\sqrt{z}$, khi đó:

\[P\geq2(\frac{a}{b+2c}+\frac{b}{c+2a}+\frac{c}{a+2b})\]
đặt $m=b+2c,n=c+2a,p=a+2b$, khi đó:

\[a=\frac{-2m + 4n + p}{9}, b=\frac{m - 2n + 4p}{9}, c=\frac{4m + n - 2p}{9}\]
\[P\geq\frac{2}{9}.(\frac{-2m + 4n + p}{m}+\frac{m - 2n + 4p}{n}+\frac{4m + n - 2p}{p})\]
\[=\frac{2}{9}.(\frac{p}{m}+\frac{m}{n}+\frac{n}{p}+4(\frac{n}{m}+\frac{p}{n}+\frac{m}{p})-6)\geq2\]
vậy giá trị nhỏ nhất của P là 2.

\section*{B}

câu 4
\newline
2) cho các số dương $x,y,z$, tìm giá trị nhỏ nhất

\[x(\frac{x}{2}+\frac{1}{yz}) + y(\frac{y}{2}+\frac{1}{zx}) + z(\frac{z}{2}+\frac{1}{xy})\]
lời giải:

\[x(\frac{x}{2}+\frac{1}{yz}) = x^2(\frac{1}{2}+\frac{1}{xyz})\]
\[\Rightarrow P=(\frac{1}{2}+\frac{1}{xyz})(x^2+y^2+z^2)\geq (\frac{1}{2}+\frac{1}{xyz}).3\sqrt[3]{x^2y^2z^2}\]
đặt $t=\sqrt[3]{xyz}$, ta được:

\[P\geq 3t^2(\frac{1}{2}+\frac{1}{t^3})=\frac{3}{2}(t^2+\frac{1}{t}+\frac{1}{t})\geq\frac{9}{2}\]
\section*{D}


câu 4
\newline
2) cho $a\geq b >0$, chứng minh rằng:

\[(2^a+\frac{1}{2^a})^b \leq (2^b+\frac{1}{2^b})^a \]
lời giải: 

\[(2^a+\frac{1}{2^a})^b \leq (2^b+\frac{1}{2^b})^a \]
\[\Leftrightarrow \frac{1}{a}.ln(2^a+\frac{1}{2^a}) \leq \frac{1}{b}.ln(2^b+\frac{1}{2^b})\]
đặt $f(x)=\frac{1}{x}.ln(2^x+\frac{1}{2^x})$
đặt $u=\frac{1}{x}, v=ln(2^x+\frac{1}{2^x})$
\[f'(x) = (uv)'=u'v + uv'\]

\[u'v=-\frac{1}{x^2}.ln(2^x+\frac{1}{2^x})\leq-\frac{1}{x^2}.ln(2^x)=-\frac{ln(2)}{x}\]
đặt $w=2^x+\frac{1}{2^x}$
\[w'=ln(2).2^x-\frac{ln(2)}{2^x}\]
\[ln(w)'=\frac{w'}{w}=ln(2).\frac{4^x-1}{4^x+1}\]
\[uv'=\frac{ln(2)}{x}.\frac{4^x-1}{4^x+1}\leq \frac{ln(2)}{x}\]

vậy $f'(x)=u'v + uv' \leq 0 \Rightarrow f(a)\leq f(b)$ 


\section*{2008}
\section*{A}
\section*{B}

câu 4
\newline
2) cho $x,y$ thỏa $x^2+y^2=1$. tìm min,max

\[\frac{2(x^2+6xy)}{1+2xy+2y^2}\]
lời giải:

\[\frac{2(x^2+6xy)}{1+2xy+2y^2} = \frac{2(x^2+6xy)}{x^2+2xy+3y^2}\]
giả sử $\frac{2(x^2+6xy)}{x^2+2xy+3y^2}\leq k$
\[\Rightarrow (k-2)x^2+3ky^2+(2k-12)xy\geq 0\]
tìm k để BDT trên dạng $(a-b)^2\geq0$, ta được $k=3$
vậy max là 3

giả sử $\frac{2(x^2+6xy)}{x^2+2xy+3y^2}\geq k$
\[\Rightarrow (2-k)x^2-3ky^2+(12-2k)xy\geq 0\]
tìm k để BDT trên dạng $(a+b)^2\geq0$, ta được $k=-6$
vậy min là -6

\section*{D}

câu 4
\newline
2) cho hai số không âm $x,y$, tìm giá trị lớn nhất và giá trị nhỏ nhất của:

\[\frac{(x-y)(1-xy)}{(1+x)^2(1+y)^2}\]
lời giải: đặt $a=1+x,b=1+y$

\[P=\frac{(a-b)(a+b-ab)}{a^2b^2}=\frac{1}{b^2}-\frac{1}{b}-(\frac{1}{a^2}-\frac{1}{a})\]
xét hàm số $f(x)=x^2-x$ trên $(0,1]$ thì $-\frac{1}{4}\leq f(x) \leq 0$

\[\Rightarrow -\frac{1}{4}\leq f(\frac{1}{b})-f(\frac{1}{a})\leq \frac{1}{4}\]


\section*{2009}
\section*{A}

câu 5: cho các số dương $x,y,z$ thỏa $x(x+y+z)=3yz$, chứng minh rằng:
\[(x+y)^3+(x+z)^3 +3(x+y)(x+z)(y+z)\leq 5(y+z)^3\]

lời giải:

\[(x+y)^3+(x+z)^3 = (2x+y+z)(2x^2+y^2+z^2 + 2xy+2xz - (x^2+xy+xz+yz))\]
\[=(2x+y+z)(y^2+z^2 +x^2+xy+xz - yz) = (2x+y+z)(y+z)^2\]

\[\Rightarrow (x+y)^3+(x+z)^3 +3(x+y)(x+z)(y+z)\leq 5(y+z)^3\]
\[\Leftrightarrow (2x+y+z)(y+z)^2 +3(x+y)(x+z)(y+z)\leq 5(y+z)^3\]
\[\Leftrightarrow (2x+y+z)(y+z) +3(x+y)(x+z)\leq 5(y+z)^2\]
\[\Leftrightarrow 2x(y+z) + (y+z)^2 + 3(x^2+xy+xz + yz)\leq 5(y+z)^2\]
\[\Leftrightarrow x(y+z) + 6yz\leq 2(y+z)^2\]
\[\Leftrightarrow 5yz\leq 2(y^2+z^2) + x^2 (*)\]

đặt $a=\frac{y}{x},b=\frac{z}{x},c=ab$, khi đó:
\[3c=1+a+b \Rightarrow (3c-1)^2=(a+b)^2\geq 4c \Leftrightarrow t\geq1 || t\leq \frac{1}{9} (**)\]

\[(*) \Leftrightarrow 5c \leq 2(a^2+b^2) + 1 = 2((a+b)^2-2c)+1 = 2((3c-1)^2-2c)+1\]
\[=18c^2-16c+3 \Leftrightarrow 6c^2-7c+1\geq0 \Leftrightarrow (c-1)(6c-1)\geq 0 (***)\]
từ $(**) \Rightarrow (***)$

\section*{B}

câu 5: cho các số thực $x,y$ thỏa $(x+y)^3+4xy \geq 2$, tìm giá trị nhỏ nhất

\[3(x^4+y^4+x^2y^2)-2(x^2+y^2)+1\]

lời giải: 

\[(x+y)^2\geq 4xy \Rightarrow (x+y)^3 + (x+y)^2 \geq (x+y)^3+4xy \geq 2\]
khảo sát hàm số $f(x)=x^3 + x^2 - 2$, ta thấy $f(x)\geq 0 \Leftrightarrow x\geq 1$
\newline
vậy $x+y \geq 1$
\[x^2+y^2+\frac{1}{2} \geq x+y \geq 1 \Rightarrow x^2+y^2 \geq \frac{1}{2}\]

\[x^4+y^4+x^2y^2 \geq \frac{3}{4}.(x^2+y^2)^2 \Rightarrow P \geq \frac{9}{4}.(x^2+y^2)^2-2(x^2+y^2)+1\]
khảo sát hàm số $f(x)=\frac{9}{4}.x^2-2x+1$, ta thấy hàm số đồng biến khi $x\geq \frac{4}{9} <\frac{1}{2}$
\newline
vậy $f(x)$ đạt giá trị nhỏ nhất khi $x=\frac{1}{2}$, giá trị đó là $\frac{9}{16}$

\section*{D}

câu 5: cho các số thực không âm $x,y$ thỏa $x+y=1$. Tìm giá trị lớn nhất và nhỏ nhất
\[(4x^2+3y)(4y^2+3x)+25xy\]

lời giải:
\[P=(4x^2+3y)(4y^2+3x)+25xy = 16x^2y^2+9xy+12(x^3+y^3) + 25xy\]
\[=16x^2y^2+34xy + 12(x+y)(x^2+y^2-xy)=16x^2y^2+34xy + 12(1-3xy)\]
\[=16x^2y^2-2xy+12\]
đặt $t=xy$, khi đó $0\leq t \leq \frac{1}{4}$
\newline
khảo sát hàm số $f(x)=16x^2-2x+12$ trên $[0,\frac{1}{4}]$, ta thấy min khi $x=\frac{1}{16}$, max khi $x=\frac{1}{4}$
\newline
vậy min là $\frac{191}{16}$, max là $\frac{25}{2}$

\section*{2010}
\section*{A}


\section*{B}
câu 5: cho các số không âm $a,b,c$ thỏa $a+b+c=1$, tìm giá trị nhỏ nhất
\[3(a^2b^2 + b^2c^2 + c^2a^2) + 3(ab+bc+ca) + 2\sqrt{a^2+b^2+c^2}\]

lời giải:

đặt $x=ab+bc+ca$, khi đó $0\leq x \leq \frac{1}{3}$

\[P \geq 3(ab+bc+ca) + 2\sqrt{a^2+b^2+c^2} = 3t+2\sqrt{1-2t}\]
khảo sát hàm số ta thấy $min=2$
\section*{D}

câu 5: tìm giá trị nhỏ nhất của hàm số
\[\sqrt{-x^2+4x+21}-\sqrt{-x^2+3x+10}\]

lời giải:









\[f'(x)= \frac{-2x+4}{2\sqrt{-x^2+4x+21}} - \frac{-2x+3}{2\sqrt{-x^2+3x+10}} \]



\section*{2011}
\section*{A}
\section*{B}
\section*{D}



\section*{2012}
\section*{A1}
\section*{A}
\section*{B}
\section*{D}


\end{document}