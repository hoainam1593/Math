\documentclass{article}
\begin{document}

\section*{2003}
\section*{A}

câu 5: cho $x,y,z$ là các số thực dương thỏa mãn $x+y+z\leq1$. Chứng minh rằng:
\[\sqrt{x^2+\frac{1}{x^2}}+\sqrt{y^2+\frac{1}{y^2}}+\sqrt{z^2+\frac{1}{z^2}}\geq\sqrt{82}\]
lời giải: 
\[x^2+\frac{1}{81x^2}\geq\frac{2}{9}\Rightarrow x^2+\frac{1}{x^2}\geq\frac{2}{9}+\frac{80}{81x^2}\]
\[\Rightarrow \sqrt{x^2+\frac{1}{x^2}}\geq\sqrt{\frac{2}{9}+\frac{80}{81x^2}}\]

dùng bất đẳng thức Bunyakovsky:
\[(a_1^2+a_2^2+...+a_n^2).(b_1^2+b_2^2+...+b_n^2)\geq(a_1.b_1+a_2.b_2+...+a_n.b_n)^2\]
cho $b_1=1, b_2=1, ..., b_n=1$ ta có:
\[n.(a_1^2+a_2^2+...+a_n^2)\geq(a_1+a_2+...+a_n)^2\]
\[\Rightarrow n.(t_1+t_2+...+t_n)\geq(\sqrt{t_1}+\sqrt{t_2}+...+\sqrt{t_n})^2\]
\[\Rightarrow \sqrt{t_1+t_2+...+t_n}\geq\frac{1}{\sqrt{n}}.(\sqrt{t_1}+\sqrt{t_2}+...+\sqrt{t_n})\]

áp dụng bất đẳng thức trên cho $\frac{2}{9}, \frac{2}{81x^2},..., \frac{2}{81x^2}$, ta có:
\[\sqrt{\frac{2}{9}+\frac{80}{81x^2}}\geq\frac{1}{\sqrt{41}}.(\sqrt{\frac{2}{9}}+40.\sqrt{\frac{2}{81x^2}})\]
\[=\frac{1}{\sqrt{41}}.(\frac{\sqrt{2}}{3}+\frac{40\sqrt{2}}{9}.\frac{1}{x})\]
tương tự cho $y,z$, từ đó ta có:
\[P\geq\frac{1}{\sqrt{41}}.(\sqrt{2}+\frac{40\sqrt{2}}{9}.(\frac{1}{x}+\frac{1}{y}+\frac{1}{z}))\]
\[\geq\frac{1}{\sqrt{41}}.(\sqrt{2}+\frac{40\sqrt{2}}{9}.\frac{9}{x+y+z})\geq\sqrt{82}\]

\section*{B}






câu 4:
\newline
1) tìm min, max của hàm số 
\[x+\sqrt{4-x^2}\]

lời giải:
đặt $y=\sqrt{4-x^2}$, khi đó $x\in [ -2,2] , y\in [ 0,2], x^2+y^2=4, f=x+y$
\[f^2=4+2xy\leq 8\Rightarrow f\leq2\sqrt{2}\]

vậy max của f la $2\sqrt{2}$
\newline
giả sử $x\leq0$, khi đó
\[f^2=4+2xy\leq4\]
suy ra min của f là -2 khi $x\leq0$, 
\newline
với $x\geq0$, dễ thấy f luôn lớn hơn 0
\newline
vậy min của f là -2


\section*{D}

\section*{2005}
\section*{A}

câu 5: cho $x,y,z$ là các số dương thỏa mãn $\frac{1}{x}+\frac{1}{y}+\frac{1}{z}=4$, chứng minh rằng:
\[\frac{1}{2x+y+z}+\frac{1}{x+2y+z}+\frac{1}{x+y+2z}\leq1\]

lời giải: áp dụng
\[\frac{1}{a+b+c+d}\leq\frac{1}{16}.(\frac{1}{a}+\frac{1}{b}+\frac{1}{c}+\frac{1}{d})\]
ta có:
\[\frac{1}{2x+y+z}\leq\frac{1}{16}.(\frac{2}{x}+\frac{1}{y}+\frac{1}{z})\]
tương tự cho $y,z$ và cộng từng vế ta được:
\[P\leq\frac{1}{4}.(\frac{1}{x}+\frac{1}{y}+\frac{1}{z})=1\]

\section*{B}


câu 5: chứng minh rằng với mọi $x\in R $ ta có:

\[(\frac{12}{5})^x+(\frac{15}{4})^x+(\frac{20}{3})^x\geq3^x+4^x+5^x\]

lời giải: áp dụng






\[\frac{ab}{c}+\frac{bc}{a}+\frac{ca}{b}\geq a+b+c\]

cho $3^x, 4^x, 5^x$

\section*{D}

câu 5: cho các số dương $x,y,z$ thỏa $xyz=1$, chứng minh rằng:

\[\frac{\sqrt{1+x^3+y^3}}{xy}+\frac{\sqrt{1+y^3+z^3}}{yz}+\frac{\sqrt{1+z^3+x^3}}{zx}\geq 3\sqrt{3}\]

lời giải:


\[1+x^3+y^3\geq3xy\Rightarrow \frac{\sqrt{1+x^3+y^3}}{xy}\geq \frac{\sqrt{3}}{\sqrt{xy}}\]
\[\Rightarrow P\geq \sqrt{3}.(\frac{1}{\sqrt{xy}}+\frac{1}{\sqrt{yz}}+\frac{1}{\sqrt{zx}})\geq3\sqrt{3}\]

\section*{2006}
\section*{A}

câu 4:
\newline
2) cho hai số thực $x,y$ thỏa







\[(x+y)xy=x^2+y^2-xy\]
tìm giá trị lớn nhất của $\frac{1}{x^3}+\frac{1}{y^3}$

lời giải:







đặt $a=\frac{1}{x}, b=\frac{1}{y}$, khi đó:






\[(\frac{1}{a}+\frac{1}{b}).\frac{1}{ab}=\frac{1}{a^2}+\frac{1}{b^2}-\frac{1}{ab}\]
\[\Rightarrow a+b=a^2+b^2-ab\Rightarrow a+b+3ab=(a+b)^2\]

đặt $t=a+b$, khi đó $ab=\frac{t^2-t}{3}$
ta có $(a+b)^2\geq4ab\Rightarrow t^2\geq\frac{4}{3}.(t^2-t)$
\[\Rightarrow 0\leq t \leq 4 \Rightarrow 0\leq a+b \leq 4\]

vậy $P=a^3+b^3=(a^2+b^2)(a+b)-ab(a+b)=(a+b+ab)(a+b)-ab(a+b)=(a+b)^2\leq16$

\section*{B}


câu 4:
\newline
2) cho số thực $x,y$ tìm giá trị nhỏ nhất của








\[\sqrt{(x-1)^2+y^2}+\sqrt{(x+1)^2+y^2}+|y-2| \]

lời giải: áp dụng BĐT Minkowski
\[\sqrt{a^2+b^2}+\sqrt{c^2+d^2}\geq\sqrt{(a+c)^2+(b+d)^2}\]
dấu = xảy ra khi $ad=bd$






\[\sqrt{(x-1)^2+y^2}+\sqrt{(x+1)^2+y^2}\geq 2\sqrt{}\]


\section*{D}

\section*{2012}
\section*{A1}
\section*{A}
\section*{B}
\section*{D}


\end{document}