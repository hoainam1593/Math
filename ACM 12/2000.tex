\documentclass{article}
\begin{document}

1. năm 2001, Mỹ sẽ là chủ nhà của olympic toán quốc tế (IMO). 
Giả sử I, M, O là các số nguyên dương khác nhau thỏa mãn $I\times M \times O=2001$. 
Giá trị lớn nhất của $I+M+O$ là bao nhiêu?

-------------------separator-------------------

1. tấc cả ước số của 2001 là
\[\{1, 3, 23, 29, 3\times 23, 23\times 29, 29\times 3, 2001\}\]
I,M,O không thể là 2001 được và không thể có hai số 1 nên ít nhất hai số phải thuộc tập hợp:
\[\{3, 23, 29, 3\times 23, 23\times 29, 29\times 3\}\]
do hai số đó không thể đồng thời nằm trong 
\[\{3\times 23, 23\times 29, 29\times 3\}\]
nên một trong ba số phải là 3, 23 hoặc 29
\newline
không mất tính tổng quát, giả sử $I=3$, khi đó $M\times O=23\times 29$.
dễ thấy M,O chỉ có thể là 23,29 hoặc $23\times 29$,1. mặt khác ta có
\[(a-1)(b-1)\geq 0 \Leftrightarrow ab+1 \geq a+b\]
vậy khi $I=3$, giá trị lớn nhất của S là $3+23\times 29 + 1$
vậy giá trị lớn nhất của S là số lớn nhất trong ba số:
\[3+23\times 29 + 1=671, 23 + 3 \times 29+1= 111, 29 + 3 \times 23+1=99\]
chính là 671.

\end{document}