\documentclass{article}
\begin{document}

1. năm 2001, Mỹ sẽ là chủ nhà của olympic toán quốc tế (IMO). 
Giả sử I, M, O là các số nguyên dương khác nhau thỏa mãn $I\times M \times O=2001$. 
Giá trị lớn nhất của $I+M+O$ là bao nhiêu?

2. rút gọn $2000(2000^{2000})$

3. mỗi ngày, Jenny ăn 20\% kẹo jellybean trong hũ vào đầu ngày. cuối ngày thứ hai, còn 32 viên kẹo.
có bao nhiêu viên kẹo lúc ban đầu.

4. dãy số Fibonacci $1,1,2,3,5,8,13,21,...$ là dãy số bắt đầu với hai số 1, sau đó mỗi phần tử là tổng của hai số trước đó.
các chữ số từ 0 đến 9 đều có thể là chữ số hàng đơn vị của các phần tử, hỏi chữ số nào xuất hiện cuối cùng khi các phần tử tăng dần?

5. cho $x<2$ và $|x-2|=p$, tính $x-p$

6. chọn hai số nguyên tố khác nhau giữa 4 và 18. lấy tích của chúng trừ đi tổng sẽ được số nào sau đây:
\[\{22,60,119,180,231\}\]

7. có bao nhiêu số nguyên dương b để cho $\log_{b}{729}$ cũng là số nguyên dương?

-------------------separator-------------------

1. tấc cả ước số của 2001 là
\[\{1, 3, 23, 29, 3\times 23, 23\times 29, 29\times 3, 2001\}\]
I,M,O không thể là 2001 được và không thể có hai số 1 nên ít nhất hai số phải thuộc tập hợp:
\[\{3, 23, 29, 3\times 23, 23\times 29, 29\times 3\}\]
do hai số đó không thể đồng thời nằm trong 
\[\{3\times 23, 23\times 29, 29\times 3\}\]
nên một trong ba số phải là 3, 23 hoặc 29
\newline
không mất tính tổng quát, giả sử $I=3$, khi đó $M\times O=23\times 29$.
dễ thấy M,O chỉ có thể là 23,29 hoặc $23\times 29$,1. mặt khác ta có
\[(a-1)(b-1)\geq 0 \Leftrightarrow ab+1 \geq a+b\]
vậy khi $I=3$, giá trị lớn nhất của S là $3+23\times 29 + 1$
vậy giá trị lớn nhất của S là số lớn nhất trong ba số:
\[3+23\times 29 + 1=671, 23 + 3 \times 29+1= 111, 29 + 3 \times 23+1=99\]
chính là 671.

2. kết quả là $2000^{2001}$

3. gọi số kẹo ban đầu là x. ăn hai lần vậy còn lại $x.\frac{4}{5}.\frac{4}{5}=32 \Rightarrow x=50$

4. ta có thể liệt kê các chữ số hàng đơn vị:
\[1,1,2,3,5,8,3,1,4,5,9,4,3,7,0,7,7,4,1,5,6\]
ta có thể thấy chữ số cuối cùng là 6

5. ta có:
\[2-x=p \Leftrightarrow x-p=2-2p\]

6. gọi hai số đó là a,b. ta có:
\[a,b \in \{5, 7, 11, 13, 17\}\]
\[S=ab-a-b=(a-1)(b-1)-1\]
đặt $x=a-1, y=b-1$, khi đó:
\[x,y \in \{4, 6, 10, 12, 16\}\]
\[S=xy-1\]
ta kiểm tra các số $\{22, 60, 119, 180, 231\}$ có thể là $xy-1$ hay không tức là 
các số $\{23, 61, 120, 181, 232\}$ có thể là $xy$ hay không
\newline
$\{23, 61, 181\}$ không thể vì xy là số chẵn
\newline
120 có thể vì $120=10 \times 12$
\newline
232 không thể vì 232 chia hết cho 29 nhưng xy thì không
\newline
vậy đáp án ban đầu là $\{119\}$













7. đặt $x=\log_{b}{729}$, khi đó:
\[b^x=729 \Rightarrow b^x=3^6\]
dựa vào bổ đề 1, ta thấy b chỉ có một ước số nguyên tố duy nhất là 3, do đó $b=3^a$, khi đó
\[3^{ax}=3^6 \Rightarrow ax=6\]
do đó a chỉ có thể là $\{1, 2, 3, 6\}$, khi đó b có thể có 4 giá trị

-------------------separator-------------------

bổ đề 1: cho p là số nguyên tố và $3^n$ chia hết cho p, chứng minh rằng $p=3$
Chứng minh: giả sử $p\neq 3$, khi đó $p=3k+1$ hoặc $p=3k+2$, khi đó
\[3^n=ap=3ak+a\]
hoặc
\[3^n=ap=3ak+2a\]
trong cả hai trường hợp dễ thấy a phải chia hết cho 3
chia cả hai vế cho 3 và lặp lại cuối cùng ta có được
$1=ap$ hoặc $3^n=p$, cả hai trường hợp đều vô lý, vậy $p=3$

\end{document}